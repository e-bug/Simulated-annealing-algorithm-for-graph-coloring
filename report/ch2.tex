In this section, we exhibit the behavior of our algorithm in different scenarios.\\
We firstly show how the residual energy varies with time in a 1000-node Erd\"os-Renyi graph, as asked in the project description.
The following figure is obtained choosing: \texttt{N=1000}, \texttt{q=7}, \texttt{c=5}, \texttt{$\beta$=5} and \texttt{maxIter=5000}.
\begin{figure}[h]
	\centering
	\setlength\figureheight{6cm} 		
	\setlength\figurewidth{0.8\textwidth}
	% This file was created by matlab2tikz.
% Minimal pgfplots version: 1.3
%
\definecolor{mycolor1}{rgb}{0.00000,0.44700,0.74100}%
%
\begin{tikzpicture}

\begin{axis}[%
width=0.95092\figurewidth,
height=\figureheight,
at={(0\figurewidth,0\figureheight)},
scale only axis,
xmin=0,
xmax=5000,
xlabel={step},
xmajorgrids,
ymin=0,
ymax=383,
ylabel={H(step)},
ymajorgrids
]
\addplot [color=mycolor1,solid,forget plot]
  table[row sep=crcr]{%
1	333\\
2	333\\
3	333\\
4	333\\
5	333\\
6	333\\
7	333\\
8	332\\
9	331\\
10	331\\
11	331\\
12	331\\
13	331\\
14	331\\
15	331\\
16	331\\
17	331\\
18	331\\
19	328\\
20	328\\
21	328\\
22	328\\
23	328\\
24	328\\
25	328\\
26	327\\
27	327\\
28	327\\
29	326\\
30	324\\
31	323\\
32	323\\
33	322\\
34	321\\
35	321\\
36	321\\
37	321\\
38	321\\
39	320\\
40	320\\
41	320\\
42	320\\
43	320\\
44	319\\
45	319\\
46	319\\
47	319\\
48	317\\
49	317\\
50	317\\
51	317\\
52	317\\
53	317\\
54	317\\
55	317\\
56	317\\
57	317\\
58	317\\
59	317\\
60	317\\
61	317\\
62	317\\
63	317\\
64	317\\
65	317\\
66	317\\
67	317\\
68	317\\
69	316\\
70	316\\
71	316\\
72	316\\
73	316\\
74	316\\
75	315\\
76	315\\
77	315\\
78	315\\
79	315\\
80	315\\
81	315\\
82	315\\
83	315\\
84	315\\
85	315\\
86	313\\
87	313\\
88	313\\
89	313\\
90	313\\
91	313\\
92	313\\
93	312\\
94	312\\
95	312\\
96	312\\
97	312\\
98	311\\
99	311\\
100	310\\
101	309\\
102	308\\
103	308\\
104	308\\
105	307\\
106	307\\
107	307\\
108	307\\
109	307\\
110	307\\
111	306\\
112	305\\
113	305\\
114	304\\
115	304\\
116	301\\
117	301\\
118	300\\
119	299\\
120	299\\
121	299\\
122	299\\
123	298\\
124	298\\
125	298\\
126	298\\
127	298\\
128	298\\
129	298\\
130	298\\
131	298\\
132	298\\
133	298\\
134	297\\
135	297\\
136	297\\
137	297\\
138	297\\
139	297\\
140	297\\
141	297\\
142	297\\
143	296\\
144	295\\
145	293\\
146	293\\
147	293\\
148	293\\
149	293\\
150	293\\
151	292\\
152	289\\
153	289\\
154	289\\
155	289\\
156	289\\
157	289\\
158	289\\
159	289\\
160	289\\
161	289\\
162	287\\
163	286\\
164	286\\
165	286\\
166	286\\
167	285\\
168	284\\
169	284\\
170	284\\
171	284\\
172	284\\
173	284\\
174	284\\
175	283\\
176	283\\
177	283\\
178	283\\
179	283\\
180	283\\
181	283\\
182	283\\
183	283\\
184	280\\
185	280\\
186	280\\
187	280\\
188	278\\
189	278\\
190	278\\
191	278\\
192	278\\
193	278\\
194	278\\
195	278\\
196	278\\
197	278\\
198	278\\
199	278\\
200	278\\
201	278\\
202	278\\
203	277\\
204	277\\
205	277\\
206	277\\
207	277\\
208	277\\
209	276\\
210	275\\
211	275\\
212	274\\
213	274\\
214	272\\
215	272\\
216	272\\
217	272\\
218	272\\
219	272\\
220	271\\
221	271\\
222	271\\
223	270\\
224	270\\
225	268\\
226	268\\
227	268\\
228	268\\
229	268\\
230	268\\
231	268\\
232	268\\
233	268\\
234	268\\
235	268\\
236	268\\
237	268\\
238	267\\
239	267\\
240	266\\
241	266\\
242	266\\
243	266\\
244	265\\
245	265\\
246	265\\
247	264\\
248	264\\
249	264\\
250	264\\
251	264\\
252	264\\
253	264\\
254	264\\
255	264\\
256	263\\
257	261\\
258	261\\
259	261\\
260	261\\
261	261\\
262	261\\
263	261\\
264	261\\
265	261\\
266	261\\
267	261\\
268	261\\
269	261\\
270	261\\
271	259\\
272	259\\
273	259\\
274	259\\
275	259\\
276	259\\
277	259\\
278	259\\
279	259\\
280	259\\
281	259\\
282	258\\
283	257\\
284	257\\
285	256\\
286	255\\
287	255\\
288	254\\
289	254\\
290	254\\
291	254\\
292	254\\
293	254\\
294	254\\
295	254\\
296	254\\
297	254\\
298	254\\
299	254\\
300	254\\
301	254\\
302	254\\
303	254\\
304	254\\
305	254\\
306	254\\
307	254\\
308	253\\
309	253\\
310	252\\
311	252\\
312	251\\
313	251\\
314	251\\
315	251\\
316	250\\
317	250\\
318	249\\
319	250\\
320	250\\
321	250\\
322	250\\
323	250\\
324	250\\
325	250\\
326	250\\
327	250\\
328	250\\
329	250\\
330	250\\
331	250\\
332	249\\
333	249\\
334	249\\
335	249\\
336	249\\
337	249\\
338	249\\
339	249\\
340	249\\
341	249\\
342	249\\
343	248\\
344	246\\
345	246\\
346	246\\
347	245\\
348	245\\
349	245\\
350	245\\
351	243\\
352	243\\
353	243\\
354	243\\
355	243\\
356	243\\
357	243\\
358	243\\
359	243\\
360	243\\
361	242\\
362	242\\
363	242\\
364	242\\
365	242\\
366	242\\
367	242\\
368	241\\
369	241\\
370	241\\
371	241\\
372	241\\
373	241\\
374	240\\
375	240\\
376	240\\
377	239\\
378	237\\
379	237\\
380	237\\
381	237\\
382	237\\
383	237\\
384	236\\
385	236\\
386	236\\
387	236\\
388	236\\
389	236\\
390	236\\
391	236\\
392	236\\
393	236\\
394	236\\
395	236\\
396	235\\
397	235\\
398	235\\
399	235\\
400	235\\
401	235\\
402	232\\
403	231\\
404	231\\
405	229\\
406	229\\
407	229\\
408	229\\
409	229\\
410	229\\
411	229\\
412	229\\
413	229\\
414	229\\
415	229\\
416	228\\
417	228\\
418	228\\
419	227\\
420	227\\
421	226\\
422	226\\
423	226\\
424	226\\
425	226\\
426	226\\
427	226\\
428	226\\
429	226\\
430	226\\
431	225\\
432	225\\
433	223\\
434	222\\
435	222\\
436	222\\
437	222\\
438	222\\
439	222\\
440	222\\
441	222\\
442	222\\
443	222\\
444	222\\
445	222\\
446	221\\
447	220\\
448	220\\
449	220\\
450	220\\
451	219\\
452	217\\
453	217\\
454	217\\
455	217\\
456	217\\
457	216\\
458	216\\
459	216\\
460	216\\
461	216\\
462	216\\
463	216\\
464	216\\
465	215\\
466	215\\
467	214\\
468	214\\
469	214\\
470	214\\
471	214\\
472	214\\
473	214\\
474	213\\
475	213\\
476	213\\
477	213\\
478	213\\
479	213\\
480	213\\
481	213\\
482	213\\
483	211\\
484	211\\
485	211\\
486	211\\
487	211\\
488	211\\
489	211\\
490	211\\
491	211\\
492	211\\
493	211\\
494	211\\
495	211\\
496	211\\
497	211\\
498	211\\
499	211\\
500	211\\
501	211\\
502	211\\
503	209\\
504	209\\
505	209\\
506	209\\
507	209\\
508	209\\
509	209\\
510	209\\
511	208\\
512	208\\
513	207\\
514	207\\
515	205\\
516	205\\
517	204\\
518	204\\
519	204\\
520	204\\
521	204\\
522	204\\
523	204\\
524	204\\
525	203\\
526	203\\
527	202\\
528	202\\
529	202\\
530	202\\
531	202\\
532	202\\
533	202\\
534	202\\
535	202\\
536	202\\
537	202\\
538	202\\
539	202\\
540	202\\
541	202\\
542	202\\
543	202\\
544	202\\
545	202\\
546	202\\
547	201\\
548	201\\
549	201\\
550	200\\
551	200\\
552	199\\
553	199\\
554	198\\
555	198\\
556	198\\
557	198\\
558	198\\
559	198\\
560	198\\
561	198\\
562	198\\
563	198\\
564	198\\
565	198\\
566	198\\
567	198\\
568	198\\
569	198\\
570	197\\
571	196\\
572	196\\
573	195\\
574	195\\
575	195\\
576	195\\
577	195\\
578	195\\
579	195\\
580	195\\
581	194\\
582	194\\
583	194\\
584	194\\
585	194\\
586	194\\
587	194\\
588	194\\
589	193\\
590	192\\
591	192\\
592	192\\
593	192\\
594	192\\
595	192\\
596	192\\
597	192\\
598	192\\
599	192\\
600	191\\
601	191\\
602	191\\
603	190\\
604	190\\
605	190\\
606	190\\
607	190\\
608	190\\
609	189\\
610	189\\
611	189\\
612	188\\
613	188\\
614	188\\
615	188\\
616	187\\
617	187\\
618	186\\
619	186\\
620	186\\
621	186\\
622	186\\
623	186\\
624	186\\
625	186\\
626	186\\
627	186\\
628	184\\
629	184\\
630	184\\
631	184\\
632	184\\
633	184\\
634	184\\
635	184\\
636	184\\
637	184\\
638	182\\
639	181\\
640	181\\
641	181\\
642	179\\
643	179\\
644	179\\
645	178\\
646	178\\
647	178\\
648	178\\
649	178\\
650	178\\
651	178\\
652	178\\
653	178\\
654	178\\
655	177\\
656	176\\
657	176\\
658	176\\
659	176\\
660	176\\
661	176\\
662	176\\
663	176\\
664	176\\
665	176\\
666	176\\
667	176\\
668	176\\
669	175\\
670	175\\
671	175\\
672	175\\
673	175\\
674	175\\
675	175\\
676	175\\
677	174\\
678	174\\
679	174\\
680	174\\
681	174\\
682	174\\
683	174\\
684	174\\
685	174\\
686	174\\
687	174\\
688	174\\
689	174\\
690	174\\
691	174\\
692	173\\
693	172\\
694	172\\
695	172\\
696	172\\
697	172\\
698	172\\
699	172\\
700	170\\
701	169\\
702	169\\
703	169\\
704	169\\
705	169\\
706	169\\
707	169\\
708	169\\
709	169\\
710	169\\
711	169\\
712	169\\
713	169\\
714	168\\
715	168\\
716	168\\
717	168\\
718	168\\
719	168\\
720	168\\
721	168\\
722	167\\
723	167\\
724	167\\
725	167\\
726	167\\
727	167\\
728	167\\
729	166\\
730	166\\
731	166\\
732	166\\
733	166\\
734	166\\
735	166\\
736	166\\
737	166\\
738	166\\
739	165\\
740	165\\
741	165\\
742	165\\
743	165\\
744	165\\
745	165\\
746	165\\
747	164\\
748	164\\
749	164\\
750	164\\
751	164\\
752	164\\
753	164\\
754	164\\
755	164\\
756	164\\
757	163\\
758	163\\
759	163\\
760	162\\
761	160\\
762	160\\
763	158\\
764	158\\
765	158\\
766	158\\
767	158\\
768	158\\
769	158\\
770	158\\
771	158\\
772	158\\
773	158\\
774	158\\
775	158\\
776	158\\
777	157\\
778	157\\
779	157\\
780	156\\
781	156\\
782	155\\
783	155\\
784	154\\
785	154\\
786	154\\
787	154\\
788	154\\
789	154\\
790	153\\
791	153\\
792	152\\
793	152\\
794	152\\
795	152\\
796	152\\
797	152\\
798	152\\
799	152\\
800	152\\
801	152\\
802	152\\
803	152\\
804	152\\
805	152\\
806	152\\
807	152\\
808	152\\
809	152\\
810	152\\
811	152\\
812	152\\
813	152\\
814	152\\
815	152\\
816	152\\
817	152\\
818	152\\
819	152\\
820	152\\
821	152\\
822	152\\
823	152\\
824	152\\
825	152\\
826	152\\
827	152\\
828	152\\
829	151\\
830	151\\
831	150\\
832	150\\
833	150\\
834	150\\
835	150\\
836	150\\
837	149\\
838	149\\
839	149\\
840	148\\
841	148\\
842	148\\
843	148\\
844	148\\
845	148\\
846	148\\
847	148\\
848	148\\
849	148\\
850	148\\
851	148\\
852	148\\
853	148\\
854	148\\
855	148\\
856	148\\
857	148\\
858	148\\
859	148\\
860	148\\
861	148\\
862	148\\
863	148\\
864	148\\
865	148\\
866	148\\
867	148\\
868	148\\
869	148\\
870	147\\
871	147\\
872	146\\
873	146\\
874	146\\
875	146\\
876	146\\
877	146\\
878	146\\
879	145\\
880	145\\
881	145\\
882	145\\
883	145\\
884	145\\
885	144\\
886	144\\
887	144\\
888	144\\
889	144\\
890	143\\
891	142\\
892	142\\
893	142\\
894	142\\
895	142\\
896	142\\
897	142\\
898	142\\
899	141\\
900	141\\
901	141\\
902	141\\
903	140\\
904	140\\
905	140\\
906	139\\
907	139\\
908	139\\
909	139\\
910	139\\
911	139\\
912	139\\
913	139\\
914	139\\
915	139\\
916	138\\
917	138\\
918	138\\
919	138\\
920	138\\
921	138\\
922	138\\
923	138\\
924	138\\
925	138\\
926	138\\
927	138\\
928	138\\
929	138\\
930	138\\
931	138\\
932	138\\
933	138\\
934	138\\
935	138\\
936	138\\
937	138\\
938	138\\
939	138\\
940	138\\
941	138\\
942	138\\
943	138\\
944	138\\
945	138\\
946	138\\
947	138\\
948	138\\
949	138\\
950	138\\
951	137\\
952	137\\
953	137\\
954	137\\
955	137\\
956	137\\
957	137\\
958	137\\
959	137\\
960	137\\
961	137\\
962	137\\
963	137\\
964	137\\
965	137\\
966	137\\
967	137\\
968	137\\
969	137\\
970	137\\
971	137\\
972	137\\
973	137\\
974	137\\
975	137\\
976	137\\
977	137\\
978	136\\
979	136\\
980	136\\
981	136\\
982	135\\
983	135\\
984	135\\
985	135\\
986	135\\
987	135\\
988	135\\
989	135\\
990	135\\
991	135\\
992	135\\
993	135\\
994	135\\
995	135\\
996	135\\
997	135\\
998	134\\
999	134\\
1000	134\\
1001	134\\
1002	134\\
1003	134\\
1004	134\\
1005	134\\
1006	134\\
1007	134\\
1008	134\\
1009	134\\
1010	134\\
1011	134\\
1012	134\\
1013	134\\
1014	134\\
1015	134\\
1016	134\\
1017	134\\
1018	134\\
1019	134\\
1020	134\\
1021	134\\
1022	134\\
1023	134\\
1024	134\\
1025	134\\
1026	134\\
1027	134\\
1028	134\\
1029	134\\
1030	134\\
1031	134\\
1032	134\\
1033	134\\
1034	133\\
1035	133\\
1036	133\\
1037	131\\
1038	131\\
1039	131\\
1040	131\\
1041	131\\
1042	131\\
1043	131\\
1044	131\\
1045	131\\
1046	130\\
1047	130\\
1048	130\\
1049	130\\
1050	130\\
1051	130\\
1052	129\\
1053	128\\
1054	128\\
1055	128\\
1056	128\\
1057	128\\
1058	128\\
1059	128\\
1060	128\\
1061	128\\
1062	128\\
1063	128\\
1064	127\\
1065	127\\
1066	127\\
1067	127\\
1068	127\\
1069	127\\
1070	127\\
1071	127\\
1072	127\\
1073	127\\
1074	127\\
1075	127\\
1076	127\\
1077	127\\
1078	127\\
1079	127\\
1080	127\\
1081	126\\
1082	126\\
1083	126\\
1084	126\\
1085	126\\
1086	126\\
1087	124\\
1088	124\\
1089	124\\
1090	123\\
1091	123\\
1092	123\\
1093	123\\
1094	123\\
1095	123\\
1096	123\\
1097	123\\
1098	122\\
1099	121\\
1100	120\\
1101	120\\
1102	120\\
1103	120\\
1104	120\\
1105	120\\
1106	120\\
1107	120\\
1108	120\\
1109	120\\
1110	120\\
1111	120\\
1112	120\\
1113	120\\
1114	120\\
1115	120\\
1116	120\\
1117	119\\
1118	118\\
1119	117\\
1120	117\\
1121	117\\
1122	117\\
1123	117\\
1124	117\\
1125	117\\
1126	117\\
1127	117\\
1128	117\\
1129	117\\
1130	117\\
1131	116\\
1132	116\\
1133	116\\
1134	116\\
1135	116\\
1136	116\\
1137	116\\
1138	115\\
1139	115\\
1140	115\\
1141	115\\
1142	115\\
1143	115\\
1144	115\\
1145	115\\
1146	115\\
1147	114\\
1148	112\\
1149	112\\
1150	112\\
1151	112\\
1152	112\\
1153	112\\
1154	112\\
1155	112\\
1156	111\\
1157	111\\
1158	111\\
1159	111\\
1160	111\\
1161	111\\
1162	111\\
1163	111\\
1164	111\\
1165	111\\
1166	111\\
1167	111\\
1168	111\\
1169	111\\
1170	111\\
1171	111\\
1172	110\\
1173	110\\
1174	110\\
1175	110\\
1176	110\\
1177	110\\
1178	110\\
1179	110\\
1180	110\\
1181	110\\
1182	110\\
1183	110\\
1184	110\\
1185	109\\
1186	109\\
1187	109\\
1188	109\\
1189	109\\
1190	109\\
1191	109\\
1192	109\\
1193	109\\
1194	109\\
1195	109\\
1196	109\\
1197	109\\
1198	109\\
1199	109\\
1200	109\\
1201	107\\
1202	107\\
1203	107\\
1204	107\\
1205	107\\
1206	107\\
1207	107\\
1208	107\\
1209	107\\
1210	107\\
1211	107\\
1212	107\\
1213	107\\
1214	106\\
1215	106\\
1216	106\\
1217	106\\
1218	106\\
1219	106\\
1220	107\\
1221	107\\
1222	107\\
1223	107\\
1224	107\\
1225	107\\
1226	107\\
1227	107\\
1228	107\\
1229	107\\
1230	107\\
1231	107\\
1232	107\\
1233	107\\
1234	107\\
1235	107\\
1236	107\\
1237	107\\
1238	107\\
1239	107\\
1240	107\\
1241	107\\
1242	107\\
1243	107\\
1244	107\\
1245	107\\
1246	107\\
1247	106\\
1248	106\\
1249	106\\
1250	106\\
1251	106\\
1252	106\\
1253	106\\
1254	106\\
1255	105\\
1256	105\\
1257	105\\
1258	105\\
1259	105\\
1260	105\\
1261	105\\
1262	105\\
1263	104\\
1264	104\\
1265	104\\
1266	104\\
1267	104\\
1268	104\\
1269	104\\
1270	104\\
1271	104\\
1272	104\\
1273	104\\
1274	104\\
1275	104\\
1276	104\\
1277	103\\
1278	103\\
1279	103\\
1280	102\\
1281	102\\
1282	102\\
1283	102\\
1284	102\\
1285	102\\
1286	102\\
1287	102\\
1288	102\\
1289	102\\
1290	102\\
1291	102\\
1292	102\\
1293	102\\
1294	102\\
1295	102\\
1296	102\\
1297	102\\
1298	102\\
1299	102\\
1300	102\\
1301	102\\
1302	101\\
1303	101\\
1304	101\\
1305	101\\
1306	101\\
1307	101\\
1308	101\\
1309	101\\
1310	101\\
1311	101\\
1312	101\\
1313	101\\
1314	101\\
1315	101\\
1316	101\\
1317	100\\
1318	100\\
1319	100\\
1320	100\\
1321	100\\
1322	100\\
1323	100\\
1324	100\\
1325	98\\
1326	98\\
1327	98\\
1328	98\\
1329	97\\
1330	97\\
1331	97\\
1332	97\\
1333	97\\
1334	97\\
1335	97\\
1336	97\\
1337	95\\
1338	95\\
1339	95\\
1340	95\\
1341	95\\
1342	95\\
1343	95\\
1344	95\\
1345	95\\
1346	95\\
1347	95\\
1348	95\\
1349	95\\
1350	95\\
1351	95\\
1352	95\\
1353	95\\
1354	95\\
1355	95\\
1356	95\\
1357	95\\
1358	95\\
1359	95\\
1360	95\\
1361	95\\
1362	95\\
1363	95\\
1364	95\\
1365	95\\
1366	95\\
1367	95\\
1368	95\\
1369	95\\
1370	95\\
1371	95\\
1372	95\\
1373	95\\
1374	95\\
1375	95\\
1376	95\\
1377	95\\
1378	94\\
1379	94\\
1380	94\\
1381	94\\
1382	94\\
1383	94\\
1384	93\\
1385	93\\
1386	93\\
1387	93\\
1388	93\\
1389	93\\
1390	93\\
1391	93\\
1392	93\\
1393	93\\
1394	93\\
1395	93\\
1396	93\\
1397	93\\
1398	93\\
1399	93\\
1400	93\\
1401	93\\
1402	93\\
1403	93\\
1404	93\\
1405	93\\
1406	93\\
1407	93\\
1408	93\\
1409	93\\
1410	93\\
1411	93\\
1412	93\\
1413	93\\
1414	93\\
1415	93\\
1416	93\\
1417	93\\
1418	93\\
1419	93\\
1420	93\\
1421	93\\
1422	93\\
1423	93\\
1424	92\\
1425	92\\
1426	92\\
1427	92\\
1428	92\\
1429	92\\
1430	92\\
1431	92\\
1432	92\\
1433	91\\
1434	91\\
1435	91\\
1436	91\\
1437	91\\
1438	91\\
1439	91\\
1440	91\\
1441	91\\
1442	91\\
1443	91\\
1444	91\\
1445	91\\
1446	91\\
1447	91\\
1448	91\\
1449	91\\
1450	91\\
1451	91\\
1452	91\\
1453	91\\
1454	91\\
1455	91\\
1456	91\\
1457	91\\
1458	91\\
1459	91\\
1460	91\\
1461	91\\
1462	91\\
1463	91\\
1464	91\\
1465	91\\
1466	91\\
1467	91\\
1468	90\\
1469	90\\
1470	90\\
1471	90\\
1472	90\\
1473	90\\
1474	90\\
1475	90\\
1476	90\\
1477	90\\
1478	90\\
1479	90\\
1480	90\\
1481	90\\
1482	90\\
1483	90\\
1484	90\\
1485	90\\
1486	90\\
1487	90\\
1488	90\\
1489	90\\
1490	89\\
1491	89\\
1492	89\\
1493	89\\
1494	89\\
1495	89\\
1496	89\\
1497	89\\
1498	89\\
1499	89\\
1500	89\\
1501	89\\
1502	88\\
1503	88\\
1504	88\\
1505	88\\
1506	88\\
1507	88\\
1508	88\\
1509	88\\
1510	88\\
1511	88\\
1512	88\\
1513	88\\
1514	88\\
1515	88\\
1516	88\\
1517	88\\
1518	88\\
1519	88\\
1520	88\\
1521	88\\
1522	88\\
1523	88\\
1524	88\\
1525	88\\
1526	88\\
1527	88\\
1528	88\\
1529	88\\
1530	88\\
1531	88\\
1532	88\\
1533	88\\
1534	88\\
1535	88\\
1536	88\\
1537	88\\
1538	88\\
1539	88\\
1540	88\\
1541	88\\
1542	88\\
1543	88\\
1544	88\\
1545	88\\
1546	88\\
1547	88\\
1548	88\\
1549	88\\
1550	88\\
1551	88\\
1552	87\\
1553	87\\
1554	87\\
1555	87\\
1556	86\\
1557	86\\
1558	86\\
1559	86\\
1560	86\\
1561	86\\
1562	86\\
1563	86\\
1564	86\\
1565	85\\
1566	84\\
1567	84\\
1568	84\\
1569	84\\
1570	84\\
1571	84\\
1572	84\\
1573	84\\
1574	84\\
1575	84\\
1576	84\\
1577	84\\
1578	84\\
1579	84\\
1580	84\\
1581	84\\
1582	84\\
1583	84\\
1584	84\\
1585	84\\
1586	84\\
1587	84\\
1588	83\\
1589	83\\
1590	82\\
1591	82\\
1592	82\\
1593	82\\
1594	82\\
1595	82\\
1596	82\\
1597	82\\
1598	82\\
1599	82\\
1600	82\\
1601	82\\
1602	82\\
1603	82\\
1604	82\\
1605	82\\
1606	82\\
1607	81\\
1608	81\\
1609	81\\
1610	81\\
1611	81\\
1612	81\\
1613	81\\
1614	81\\
1615	81\\
1616	81\\
1617	81\\
1618	81\\
1619	81\\
1620	81\\
1621	81\\
1622	81\\
1623	81\\
1624	81\\
1625	81\\
1626	81\\
1627	81\\
1628	81\\
1629	81\\
1630	81\\
1631	81\\
1632	81\\
1633	81\\
1634	81\\
1635	81\\
1636	80\\
1637	80\\
1638	80\\
1639	80\\
1640	80\\
1641	80\\
1642	80\\
1643	80\\
1644	80\\
1645	80\\
1646	80\\
1647	80\\
1648	80\\
1649	80\\
1650	80\\
1651	80\\
1652	80\\
1653	80\\
1654	80\\
1655	80\\
1656	80\\
1657	80\\
1658	80\\
1659	80\\
1660	79\\
1661	79\\
1662	79\\
1663	79\\
1664	79\\
1665	79\\
1666	79\\
1667	79\\
1668	79\\
1669	79\\
1670	79\\
1671	79\\
1672	80\\
1673	80\\
1674	80\\
1675	80\\
1676	80\\
1677	80\\
1678	80\\
1679	80\\
1680	80\\
1681	80\\
1682	80\\
1683	80\\
1684	80\\
1685	80\\
1686	80\\
1687	80\\
1688	80\\
1689	80\\
1690	80\\
1691	80\\
1692	80\\
1693	80\\
1694	80\\
1695	79\\
1696	79\\
1697	79\\
1698	79\\
1699	79\\
1700	79\\
1701	79\\
1702	79\\
1703	79\\
1704	79\\
1705	79\\
1706	79\\
1707	79\\
1708	79\\
1709	79\\
1710	79\\
1711	79\\
1712	79\\
1713	79\\
1714	79\\
1715	78\\
1716	78\\
1717	78\\
1718	78\\
1719	78\\
1720	78\\
1721	77\\
1722	77\\
1723	77\\
1724	77\\
1725	76\\
1726	76\\
1727	76\\
1728	76\\
1729	76\\
1730	76\\
1731	76\\
1732	76\\
1733	76\\
1734	76\\
1735	76\\
1736	76\\
1737	76\\
1738	76\\
1739	76\\
1740	76\\
1741	76\\
1742	76\\
1743	76\\
1744	75\\
1745	75\\
1746	75\\
1747	75\\
1748	75\\
1749	75\\
1750	75\\
1751	75\\
1752	75\\
1753	75\\
1754	75\\
1755	75\\
1756	75\\
1757	75\\
1758	75\\
1759	75\\
1760	75\\
1761	75\\
1762	75\\
1763	75\\
1764	75\\
1765	75\\
1766	75\\
1767	75\\
1768	75\\
1769	75\\
1770	75\\
1771	75\\
1772	75\\
1773	75\\
1774	75\\
1775	75\\
1776	75\\
1777	75\\
1778	75\\
1779	75\\
1780	75\\
1781	75\\
1782	75\\
1783	75\\
1784	74\\
1785	74\\
1786	74\\
1787	74\\
1788	74\\
1789	74\\
1790	74\\
1791	74\\
1792	74\\
1793	74\\
1794	74\\
1795	74\\
1796	74\\
1797	74\\
1798	74\\
1799	74\\
1800	74\\
1801	74\\
1802	74\\
1803	74\\
1804	74\\
1805	74\\
1806	74\\
1807	74\\
1808	74\\
1809	74\\
1810	74\\
1811	73\\
1812	73\\
1813	73\\
1814	73\\
1815	73\\
1816	73\\
1817	73\\
1818	73\\
1819	73\\
1820	73\\
1821	73\\
1822	73\\
1823	73\\
1824	73\\
1825	73\\
1826	73\\
1827	73\\
1828	73\\
1829	73\\
1830	73\\
1831	73\\
1832	73\\
1833	72\\
1834	72\\
1835	72\\
1836	72\\
1837	72\\
1838	72\\
1839	72\\
1840	72\\
1841	72\\
1842	72\\
1843	72\\
1844	72\\
1845	72\\
1846	72\\
1847	72\\
1848	72\\
1849	72\\
1850	72\\
1851	71\\
1852	71\\
1853	71\\
1854	71\\
1855	71\\
1856	71\\
1857	71\\
1858	71\\
1859	71\\
1860	71\\
1861	71\\
1862	71\\
1863	71\\
1864	71\\
1865	71\\
1866	71\\
1867	71\\
1868	71\\
1869	71\\
1870	71\\
1871	71\\
1872	71\\
1873	70\\
1874	70\\
1875	70\\
1876	70\\
1877	70\\
1878	70\\
1879	70\\
1880	70\\
1881	70\\
1882	70\\
1883	70\\
1884	70\\
1885	70\\
1886	70\\
1887	70\\
1888	70\\
1889	70\\
1890	70\\
1891	70\\
1892	70\\
1893	70\\
1894	70\\
1895	70\\
1896	70\\
1897	70\\
1898	70\\
1899	70\\
1900	70\\
1901	70\\
1902	70\\
1903	70\\
1904	70\\
1905	70\\
1906	70\\
1907	70\\
1908	70\\
1909	70\\
1910	70\\
1911	70\\
1912	70\\
1913	70\\
1914	70\\
1915	70\\
1916	70\\
1917	70\\
1918	70\\
1919	70\\
1920	70\\
1921	70\\
1922	70\\
1923	70\\
1924	70\\
1925	70\\
1926	69\\
1927	69\\
1928	68\\
1929	68\\
1930	68\\
1931	68\\
1932	68\\
1933	68\\
1934	68\\
1935	67\\
1936	66\\
1937	66\\
1938	66\\
1939	66\\
1940	66\\
1941	66\\
1942	66\\
1943	66\\
1944	66\\
1945	66\\
1946	66\\
1947	66\\
1948	66\\
1949	66\\
1950	66\\
1951	66\\
1952	66\\
1953	66\\
1954	66\\
1955	66\\
1956	66\\
1957	66\\
1958	66\\
1959	66\\
1960	66\\
1961	66\\
1962	66\\
1963	66\\
1964	66\\
1965	66\\
1966	66\\
1967	66\\
1968	66\\
1969	66\\
1970	66\\
1971	66\\
1972	66\\
1973	66\\
1974	66\\
1975	66\\
1976	66\\
1977	66\\
1978	66\\
1979	66\\
1980	66\\
1981	66\\
1982	66\\
1983	66\\
1984	66\\
1985	66\\
1986	66\\
1987	66\\
1988	66\\
1989	66\\
1990	66\\
1991	66\\
1992	66\\
1993	66\\
1994	66\\
1995	66\\
1996	66\\
1997	66\\
1998	66\\
1999	65\\
2000	65\\
2001	65\\
2002	65\\
2003	65\\
2004	65\\
2005	65\\
2006	65\\
2007	65\\
2008	65\\
2009	65\\
2010	65\\
2011	65\\
2012	65\\
2013	65\\
2014	65\\
2015	64\\
2016	64\\
2017	64\\
2018	64\\
2019	64\\
2020	64\\
2021	64\\
2022	64\\
2023	64\\
2024	64\\
2025	64\\
2026	64\\
2027	64\\
2028	64\\
2029	64\\
2030	64\\
2031	64\\
2032	64\\
2033	64\\
2034	63\\
2035	63\\
2036	63\\
2037	63\\
2038	63\\
2039	63\\
2040	63\\
2041	63\\
2042	62\\
2043	62\\
2044	62\\
2045	62\\
2046	62\\
2047	62\\
2048	62\\
2049	61\\
2050	61\\
2051	61\\
2052	61\\
2053	61\\
2054	61\\
2055	61\\
2056	61\\
2057	61\\
2058	61\\
2059	60\\
2060	60\\
2061	59\\
2062	59\\
2063	59\\
2064	59\\
2065	59\\
2066	59\\
2067	59\\
2068	59\\
2069	59\\
2070	59\\
2071	59\\
2072	59\\
2073	59\\
2074	59\\
2075	59\\
2076	59\\
2077	59\\
2078	59\\
2079	59\\
2080	59\\
2081	59\\
2082	59\\
2083	59\\
2084	59\\
2085	59\\
2086	59\\
2087	59\\
2088	59\\
2089	59\\
2090	59\\
2091	59\\
2092	59\\
2093	59\\
2094	59\\
2095	59\\
2096	59\\
2097	59\\
2098	59\\
2099	59\\
2100	59\\
2101	59\\
2102	59\\
2103	59\\
2104	59\\
2105	59\\
2106	59\\
2107	59\\
2108	59\\
2109	59\\
2110	59\\
2111	59\\
2112	59\\
2113	59\\
2114	59\\
2115	59\\
2116	59\\
2117	59\\
2118	59\\
2119	59\\
2120	59\\
2121	59\\
2122	57\\
2123	57\\
2124	57\\
2125	57\\
2126	57\\
2127	57\\
2128	57\\
2129	57\\
2130	57\\
2131	57\\
2132	57\\
2133	57\\
2134	57\\
2135	57\\
2136	57\\
2137	57\\
2138	57\\
2139	57\\
2140	56\\
2141	56\\
2142	56\\
2143	56\\
2144	56\\
2145	56\\
2146	56\\
2147	56\\
2148	56\\
2149	56\\
2150	56\\
2151	56\\
2152	56\\
2153	56\\
2154	56\\
2155	56\\
2156	56\\
2157	56\\
2158	56\\
2159	56\\
2160	56\\
2161	56\\
2162	56\\
2163	56\\
2164	56\\
2165	56\\
2166	56\\
2167	56\\
2168	56\\
2169	56\\
2170	56\\
2171	56\\
2172	56\\
2173	56\\
2174	56\\
2175	56\\
2176	56\\
2177	56\\
2178	56\\
2179	56\\
2180	55\\
2181	55\\
2182	55\\
2183	55\\
2184	55\\
2185	55\\
2186	55\\
2187	55\\
2188	54\\
2189	54\\
2190	54\\
2191	53\\
2192	53\\
2193	53\\
2194	53\\
2195	53\\
2196	53\\
2197	53\\
2198	53\\
2199	53\\
2200	53\\
2201	53\\
2202	53\\
2203	53\\
2204	53\\
2205	53\\
2206	53\\
2207	53\\
2208	53\\
2209	53\\
2210	53\\
2211	53\\
2212	53\\
2213	53\\
2214	53\\
2215	53\\
2216	53\\
2217	53\\
2218	53\\
2219	53\\
2220	53\\
2221	53\\
2222	53\\
2223	53\\
2224	53\\
2225	53\\
2226	53\\
2227	53\\
2228	53\\
2229	52\\
2230	52\\
2231	52\\
2232	52\\
2233	52\\
2234	52\\
2235	52\\
2236	52\\
2237	52\\
2238	52\\
2239	52\\
2240	52\\
2241	52\\
2242	52\\
2243	52\\
2244	52\\
2245	52\\
2246	52\\
2247	52\\
2248	52\\
2249	52\\
2250	52\\
2251	52\\
2252	52\\
2253	52\\
2254	52\\
2255	52\\
2256	52\\
2257	52\\
2258	52\\
2259	52\\
2260	52\\
2261	52\\
2262	52\\
2263	52\\
2264	52\\
2265	52\\
2266	52\\
2267	52\\
2268	52\\
2269	52\\
2270	52\\
2271	52\\
2272	52\\
2273	52\\
2274	52\\
2275	52\\
2276	52\\
2277	52\\
2278	52\\
2279	52\\
2280	52\\
2281	52\\
2282	52\\
2283	52\\
2284	52\\
2285	52\\
2286	52\\
2287	52\\
2288	52\\
2289	52\\
2290	52\\
2291	52\\
2292	52\\
2293	52\\
2294	52\\
2295	52\\
2296	52\\
2297	52\\
2298	52\\
2299	52\\
2300	52\\
2301	52\\
2302	52\\
2303	52\\
2304	52\\
2305	52\\
2306	52\\
2307	52\\
2308	52\\
2309	52\\
2310	52\\
2311	52\\
2312	51\\
2313	51\\
2314	50\\
2315	50\\
2316	50\\
2317	50\\
2318	50\\
2319	50\\
2320	50\\
2321	50\\
2322	50\\
2323	50\\
2324	50\\
2325	50\\
2326	50\\
2327	50\\
2328	50\\
2329	50\\
2330	50\\
2331	50\\
2332	50\\
2333	50\\
2334	50\\
2335	50\\
2336	50\\
2337	50\\
2338	50\\
2339	50\\
2340	50\\
2341	50\\
2342	50\\
2343	50\\
2344	50\\
2345	50\\
2346	50\\
2347	50\\
2348	50\\
2349	50\\
2350	50\\
2351	50\\
2352	50\\
2353	50\\
2354	50\\
2355	50\\
2356	50\\
2357	50\\
2358	50\\
2359	50\\
2360	50\\
2361	50\\
2362	50\\
2363	50\\
2364	49\\
2365	49\\
2366	49\\
2367	49\\
2368	49\\
2369	49\\
2370	49\\
2371	49\\
2372	49\\
2373	49\\
2374	49\\
2375	49\\
2376	48\\
2377	48\\
2378	48\\
2379	48\\
2380	48\\
2381	48\\
2382	48\\
2383	48\\
2384	48\\
2385	47\\
2386	47\\
2387	47\\
2388	47\\
2389	47\\
2390	47\\
2391	47\\
2392	47\\
2393	47\\
2394	47\\
2395	47\\
2396	47\\
2397	47\\
2398	47\\
2399	47\\
2400	47\\
2401	47\\
2402	47\\
2403	47\\
2404	47\\
2405	47\\
2406	47\\
2407	47\\
2408	47\\
2409	47\\
2410	47\\
2411	47\\
2412	47\\
2413	47\\
2414	47\\
2415	47\\
2416	47\\
2417	47\\
2418	47\\
2419	47\\
2420	47\\
2421	47\\
2422	47\\
2423	47\\
2424	47\\
2425	47\\
2426	47\\
2427	47\\
2428	47\\
2429	47\\
2430	47\\
2431	47\\
2432	47\\
2433	47\\
2434	46\\
2435	45\\
2436	45\\
2437	45\\
2438	45\\
2439	45\\
2440	45\\
2441	45\\
2442	45\\
2443	45\\
2444	45\\
2445	45\\
2446	45\\
2447	44\\
2448	44\\
2449	44\\
2450	44\\
2451	44\\
2452	44\\
2453	44\\
2454	44\\
2455	44\\
2456	44\\
2457	44\\
2458	44\\
2459	44\\
2460	44\\
2461	44\\
2462	44\\
2463	44\\
2464	44\\
2465	44\\
2466	44\\
2467	44\\
2468	44\\
2469	44\\
2470	44\\
2471	44\\
2472	44\\
2473	44\\
2474	44\\
2475	44\\
2476	44\\
2477	44\\
2478	44\\
2479	44\\
2480	44\\
2481	44\\
2482	44\\
2483	44\\
2484	44\\
2485	44\\
2486	44\\
2487	44\\
2488	44\\
2489	44\\
2490	44\\
2491	44\\
2492	44\\
2493	44\\
2494	44\\
2495	44\\
2496	44\\
2497	44\\
2498	44\\
2499	44\\
2500	44\\
2501	44\\
2502	44\\
2503	44\\
2504	44\\
2505	44\\
2506	44\\
2507	44\\
2508	44\\
2509	44\\
2510	44\\
2511	44\\
2512	44\\
2513	44\\
2514	44\\
2515	44\\
2516	44\\
2517	44\\
2518	44\\
2519	44\\
2520	44\\
2521	44\\
2522	44\\
2523	44\\
2524	44\\
2525	44\\
2526	44\\
2527	44\\
2528	44\\
2529	44\\
2530	44\\
2531	44\\
2532	44\\
2533	44\\
2534	44\\
2535	43\\
2536	43\\
2537	43\\
2538	43\\
2539	43\\
2540	43\\
2541	43\\
2542	43\\
2543	43\\
2544	43\\
2545	43\\
2546	43\\
2547	43\\
2548	43\\
2549	43\\
2550	43\\
2551	43\\
2552	43\\
2553	43\\
2554	43\\
2555	43\\
2556	43\\
2557	43\\
2558	43\\
2559	43\\
2560	43\\
2561	43\\
2562	43\\
2563	43\\
2564	43\\
2565	43\\
2566	43\\
2567	43\\
2568	43\\
2569	43\\
2570	43\\
2571	43\\
2572	43\\
2573	43\\
2574	43\\
2575	43\\
2576	43\\
2577	43\\
2578	43\\
2579	43\\
2580	43\\
2581	43\\
2582	43\\
2583	43\\
2584	43\\
2585	43\\
2586	43\\
2587	43\\
2588	43\\
2589	43\\
2590	43\\
2591	43\\
2592	43\\
2593	43\\
2594	43\\
2595	43\\
2596	43\\
2597	43\\
2598	43\\
2599	43\\
2600	43\\
2601	43\\
2602	43\\
2603	43\\
2604	43\\
2605	43\\
2606	43\\
2607	43\\
2608	43\\
2609	43\\
2610	43\\
2611	43\\
2612	42\\
2613	42\\
2614	42\\
2615	42\\
2616	42\\
2617	42\\
2618	42\\
2619	42\\
2620	42\\
2621	42\\
2622	42\\
2623	42\\
2624	42\\
2625	42\\
2626	42\\
2627	42\\
2628	42\\
2629	42\\
2630	42\\
2631	42\\
2632	42\\
2633	42\\
2634	42\\
2635	42\\
2636	42\\
2637	42\\
2638	42\\
2639	41\\
2640	41\\
2641	41\\
2642	41\\
2643	41\\
2644	41\\
2645	41\\
2646	41\\
2647	41\\
2648	41\\
2649	41\\
2650	41\\
2651	41\\
2652	41\\
2653	41\\
2654	41\\
2655	41\\
2656	41\\
2657	41\\
2658	41\\
2659	41\\
2660	41\\
2661	41\\
2662	41\\
2663	41\\
2664	41\\
2665	41\\
2666	41\\
2667	41\\
2668	41\\
2669	41\\
2670	41\\
2671	41\\
2672	41\\
2673	41\\
2674	41\\
2675	41\\
2676	41\\
2677	41\\
2678	41\\
2679	41\\
2680	41\\
2681	41\\
2682	41\\
2683	41\\
2684	41\\
2685	41\\
2686	41\\
2687	41\\
2688	41\\
2689	41\\
2690	41\\
2691	41\\
2692	41\\
2693	41\\
2694	41\\
2695	41\\
2696	41\\
2697	41\\
2698	41\\
2699	41\\
2700	41\\
2701	41\\
2702	41\\
2703	41\\
2704	41\\
2705	41\\
2706	41\\
2707	41\\
2708	41\\
2709	41\\
2710	41\\
2711	41\\
2712	41\\
2713	41\\
2714	41\\
2715	41\\
2716	41\\
2717	41\\
2718	41\\
2719	41\\
2720	41\\
2721	41\\
2722	41\\
2723	41\\
2724	41\\
2725	41\\
2726	41\\
2727	41\\
2728	41\\
2729	41\\
2730	41\\
2731	41\\
2732	41\\
2733	41\\
2734	41\\
2735	41\\
2736	41\\
2737	41\\
2738	41\\
2739	40\\
2740	40\\
2741	40\\
2742	40\\
2743	40\\
2744	40\\
2745	40\\
2746	40\\
2747	40\\
2748	40\\
2749	40\\
2750	40\\
2751	40\\
2752	40\\
2753	40\\
2754	40\\
2755	40\\
2756	40\\
2757	40\\
2758	40\\
2759	40\\
2760	40\\
2761	39\\
2762	39\\
2763	39\\
2764	39\\
2765	39\\
2766	39\\
2767	39\\
2768	39\\
2769	39\\
2770	40\\
2771	40\\
2772	40\\
2773	40\\
2774	39\\
2775	39\\
2776	39\\
2777	39\\
2778	39\\
2779	39\\
2780	39\\
2781	39\\
2782	39\\
2783	39\\
2784	39\\
2785	39\\
2786	39\\
2787	39\\
2788	39\\
2789	39\\
2790	39\\
2791	39\\
2792	39\\
2793	39\\
2794	39\\
2795	39\\
2796	39\\
2797	39\\
2798	38\\
2799	38\\
2800	38\\
2801	38\\
2802	38\\
2803	38\\
2804	38\\
2805	38\\
2806	38\\
2807	38\\
2808	38\\
2809	38\\
2810	38\\
2811	38\\
2812	38\\
2813	38\\
2814	38\\
2815	38\\
2816	38\\
2817	38\\
2818	38\\
2819	38\\
2820	38\\
2821	38\\
2822	38\\
2823	38\\
2824	38\\
2825	38\\
2826	38\\
2827	38\\
2828	38\\
2829	38\\
2830	38\\
2831	38\\
2832	38\\
2833	38\\
2834	38\\
2835	36\\
2836	36\\
2837	36\\
2838	36\\
2839	35\\
2840	35\\
2841	35\\
2842	35\\
2843	35\\
2844	35\\
2845	35\\
2846	35\\
2847	35\\
2848	35\\
2849	35\\
2850	35\\
2851	35\\
2852	35\\
2853	35\\
2854	35\\
2855	35\\
2856	35\\
2857	35\\
2858	35\\
2859	35\\
2860	35\\
2861	35\\
2862	35\\
2863	35\\
2864	35\\
2865	35\\
2866	35\\
2867	35\\
2868	35\\
2869	35\\
2870	35\\
2871	35\\
2872	34\\
2873	34\\
2874	34\\
2875	34\\
2876	34\\
2877	34\\
2878	34\\
2879	33\\
2880	33\\
2881	33\\
2882	33\\
2883	33\\
2884	33\\
2885	33\\
2886	33\\
2887	32\\
2888	32\\
2889	32\\
2890	32\\
2891	32\\
2892	32\\
2893	32\\
2894	32\\
2895	32\\
2896	32\\
2897	32\\
2898	32\\
2899	32\\
2900	32\\
2901	32\\
2902	32\\
2903	32\\
2904	32\\
2905	32\\
2906	32\\
2907	32\\
2908	32\\
2909	32\\
2910	32\\
2911	32\\
2912	32\\
2913	32\\
2914	32\\
2915	32\\
2916	32\\
2917	32\\
2918	32\\
2919	32\\
2920	32\\
2921	32\\
2922	32\\
2923	32\\
2924	32\\
2925	32\\
2926	32\\
2927	32\\
2928	32\\
2929	32\\
2930	32\\
2931	32\\
2932	32\\
2933	32\\
2934	32\\
2935	32\\
2936	32\\
2937	32\\
2938	32\\
2939	32\\
2940	32\\
2941	32\\
2942	32\\
2943	32\\
2944	32\\
2945	32\\
2946	32\\
2947	32\\
2948	32\\
2949	32\\
2950	32\\
2951	32\\
2952	32\\
2953	32\\
2954	32\\
2955	32\\
2956	32\\
2957	32\\
2958	32\\
2959	32\\
2960	32\\
2961	32\\
2962	32\\
2963	32\\
2964	32\\
2965	32\\
2966	32\\
2967	32\\
2968	32\\
2969	32\\
2970	32\\
2971	32\\
2972	32\\
2973	32\\
2974	32\\
2975	32\\
2976	32\\
2977	32\\
2978	32\\
2979	32\\
2980	32\\
2981	32\\
2982	32\\
2983	32\\
2984	32\\
2985	32\\
2986	32\\
2987	32\\
2988	32\\
2989	32\\
2990	32\\
2991	32\\
2992	32\\
2993	32\\
2994	32\\
2995	32\\
2996	32\\
2997	32\\
2998	32\\
2999	32\\
3000	32\\
3001	32\\
3002	32\\
3003	32\\
3004	32\\
3005	32\\
3006	32\\
3007	32\\
3008	32\\
3009	32\\
3010	32\\
3011	32\\
3012	32\\
3013	32\\
3014	32\\
3015	32\\
3016	32\\
3017	32\\
3018	32\\
3019	31\\
3020	31\\
3021	31\\
3022	30\\
3023	30\\
3024	30\\
3025	30\\
3026	30\\
3027	30\\
3028	30\\
3029	30\\
3030	31\\
3031	31\\
3032	31\\
3033	31\\
3034	31\\
3035	31\\
3036	31\\
3037	31\\
3038	31\\
3039	31\\
3040	31\\
3041	31\\
3042	31\\
3043	31\\
3044	31\\
3045	31\\
3046	31\\
3047	31\\
3048	31\\
3049	31\\
3050	31\\
3051	31\\
3052	31\\
3053	31\\
3054	31\\
3055	31\\
3056	31\\
3057	31\\
3058	31\\
3059	31\\
3060	31\\
3061	31\\
3062	31\\
3063	31\\
3064	31\\
3065	32\\
3066	32\\
3067	32\\
3068	32\\
3069	32\\
3070	32\\
3071	32\\
3072	32\\
3073	32\\
3074	32\\
3075	32\\
3076	32\\
3077	32\\
3078	32\\
3079	32\\
3080	32\\
3081	32\\
3082	32\\
3083	32\\
3084	32\\
3085	32\\
3086	32\\
3087	32\\
3088	32\\
3089	32\\
3090	32\\
3091	32\\
3092	32\\
3093	32\\
3094	32\\
3095	32\\
3096	32\\
3097	32\\
3098	32\\
3099	32\\
3100	32\\
3101	33\\
3102	33\\
3103	33\\
3104	33\\
3105	33\\
3106	33\\
3107	33\\
3108	33\\
3109	33\\
3110	33\\
3111	33\\
3112	33\\
3113	33\\
3114	33\\
3115	33\\
3116	33\\
3117	33\\
3118	33\\
3119	33\\
3120	33\\
3121	33\\
3122	33\\
3123	33\\
3124	33\\
3125	33\\
3126	33\\
3127	33\\
3128	33\\
3129	33\\
3130	33\\
3131	33\\
3132	33\\
3133	33\\
3134	33\\
3135	33\\
3136	33\\
3137	32\\
3138	32\\
3139	32\\
3140	32\\
3141	32\\
3142	32\\
3143	31\\
3144	31\\
3145	31\\
3146	31\\
3147	31\\
3148	31\\
3149	31\\
3150	31\\
3151	31\\
3152	31\\
3153	31\\
3154	31\\
3155	31\\
3156	31\\
3157	31\\
3158	31\\
3159	31\\
3160	31\\
3161	31\\
3162	31\\
3163	31\\
3164	31\\
3165	31\\
3166	31\\
3167	31\\
3168	31\\
3169	31\\
3170	31\\
3171	31\\
3172	31\\
3173	31\\
3174	31\\
3175	31\\
3176	31\\
3177	31\\
3178	31\\
3179	31\\
3180	31\\
3181	31\\
3182	31\\
3183	31\\
3184	31\\
3185	31\\
3186	31\\
3187	31\\
3188	31\\
3189	31\\
3190	31\\
3191	31\\
3192	31\\
3193	31\\
3194	31\\
3195	31\\
3196	31\\
3197	31\\
3198	31\\
3199	30\\
3200	30\\
3201	30\\
3202	29\\
3203	29\\
3204	29\\
3205	29\\
3206	29\\
3207	29\\
3208	29\\
3209	29\\
3210	29\\
3211	29\\
3212	29\\
3213	29\\
3214	29\\
3215	29\\
3216	29\\
3217	29\\
3218	29\\
3219	29\\
3220	29\\
3221	29\\
3222	29\\
3223	29\\
3224	29\\
3225	29\\
3226	29\\
3227	29\\
3228	29\\
3229	29\\
3230	29\\
3231	29\\
3232	29\\
3233	29\\
3234	29\\
3235	29\\
3236	29\\
3237	29\\
3238	29\\
3239	29\\
3240	29\\
3241	29\\
3242	29\\
3243	29\\
3244	29\\
3245	29\\
3246	29\\
3247	29\\
3248	29\\
3249	29\\
3250	29\\
3251	29\\
3252	29\\
3253	29\\
3254	29\\
3255	29\\
3256	29\\
3257	29\\
3258	29\\
3259	29\\
3260	29\\
3261	29\\
3262	29\\
3263	29\\
3264	29\\
3265	29\\
3266	29\\
3267	29\\
3268	29\\
3269	29\\
3270	29\\
3271	29\\
3272	29\\
3273	29\\
3274	29\\
3275	29\\
3276	29\\
3277	29\\
3278	29\\
3279	29\\
3280	29\\
3281	29\\
3282	29\\
3283	29\\
3284	29\\
3285	29\\
3286	29\\
3287	29\\
3288	29\\
3289	29\\
3290	29\\
3291	29\\
3292	29\\
3293	29\\
3294	28\\
3295	28\\
3296	28\\
3297	28\\
3298	28\\
3299	28\\
3300	28\\
3301	28\\
3302	28\\
3303	28\\
3304	28\\
3305	28\\
3306	28\\
3307	28\\
3308	28\\
3309	28\\
3310	28\\
3311	28\\
3312	28\\
3313	28\\
3314	28\\
3315	28\\
3316	28\\
3317	28\\
3318	28\\
3319	28\\
3320	28\\
3321	28\\
3322	28\\
3323	28\\
3324	28\\
3325	28\\
3326	28\\
3327	28\\
3328	28\\
3329	28\\
3330	28\\
3331	28\\
3332	28\\
3333	28\\
3334	28\\
3335	28\\
3336	28\\
3337	28\\
3338	28\\
3339	28\\
3340	28\\
3341	28\\
3342	28\\
3343	28\\
3344	28\\
3345	28\\
3346	28\\
3347	28\\
3348	28\\
3349	28\\
3350	28\\
3351	28\\
3352	28\\
3353	28\\
3354	28\\
3355	28\\
3356	28\\
3357	28\\
3358	28\\
3359	28\\
3360	28\\
3361	28\\
3362	28\\
3363	28\\
3364	28\\
3365	28\\
3366	28\\
3367	28\\
3368	28\\
3369	28\\
3370	28\\
3371	28\\
3372	28\\
3373	28\\
3374	28\\
3375	28\\
3376	28\\
3377	28\\
3378	28\\
3379	28\\
3380	28\\
3381	26\\
3382	26\\
3383	26\\
3384	26\\
3385	26\\
3386	26\\
3387	26\\
3388	26\\
3389	26\\
3390	26\\
3391	26\\
3392	26\\
3393	26\\
3394	26\\
3395	26\\
3396	26\\
3397	26\\
3398	26\\
3399	26\\
3400	26\\
3401	26\\
3402	26\\
3403	25\\
3404	25\\
3405	25\\
3406	25\\
3407	25\\
3408	25\\
3409	24\\
3410	24\\
3411	24\\
3412	24\\
3413	24\\
3414	24\\
3415	24\\
3416	24\\
3417	24\\
3418	24\\
3419	24\\
3420	24\\
3421	24\\
3422	24\\
3423	24\\
3424	24\\
3425	24\\
3426	24\\
3427	24\\
3428	24\\
3429	24\\
3430	24\\
3431	24\\
3432	24\\
3433	24\\
3434	24\\
3435	24\\
3436	24\\
3437	24\\
3438	24\\
3439	24\\
3440	24\\
3441	24\\
3442	24\\
3443	24\\
3444	24\\
3445	24\\
3446	24\\
3447	24\\
3448	24\\
3449	24\\
3450	24\\
3451	24\\
3452	24\\
3453	23\\
3454	23\\
3455	23\\
3456	23\\
3457	23\\
3458	23\\
3459	23\\
3460	23\\
3461	23\\
3462	23\\
3463	23\\
3464	23\\
3465	23\\
3466	23\\
3467	23\\
3468	23\\
3469	23\\
3470	23\\
3471	23\\
3472	23\\
3473	23\\
3474	23\\
3475	23\\
3476	23\\
3477	23\\
3478	23\\
3479	23\\
3480	23\\
3481	23\\
3482	23\\
3483	23\\
3484	22\\
3485	22\\
3486	22\\
3487	22\\
3488	22\\
3489	22\\
3490	22\\
3491	22\\
3492	22\\
3493	22\\
3494	22\\
3495	22\\
3496	22\\
3497	22\\
3498	22\\
3499	22\\
3500	22\\
3501	22\\
3502	22\\
3503	22\\
3504	21\\
3505	21\\
3506	21\\
3507	21\\
3508	21\\
3509	21\\
3510	21\\
3511	21\\
3512	20\\
3513	20\\
3514	20\\
3515	20\\
3516	20\\
3517	20\\
3518	20\\
3519	20\\
3520	20\\
3521	20\\
3522	20\\
3523	20\\
3524	20\\
3525	20\\
3526	20\\
3527	20\\
3528	20\\
3529	20\\
3530	20\\
3531	20\\
3532	20\\
3533	20\\
3534	20\\
3535	20\\
3536	20\\
3537	19\\
3538	19\\
3539	19\\
3540	19\\
3541	19\\
3542	19\\
3543	19\\
3544	19\\
3545	19\\
3546	19\\
3547	19\\
3548	19\\
3549	19\\
3550	19\\
3551	19\\
3552	19\\
3553	19\\
3554	19\\
3555	19\\
3556	19\\
3557	19\\
3558	19\\
3559	19\\
3560	19\\
3561	19\\
3562	19\\
3563	19\\
3564	19\\
3565	19\\
3566	19\\
3567	19\\
3568	19\\
3569	19\\
3570	19\\
3571	19\\
3572	19\\
3573	19\\
3574	19\\
3575	19\\
3576	19\\
3577	19\\
3578	19\\
3579	19\\
3580	19\\
3581	19\\
3582	19\\
3583	19\\
3584	19\\
3585	19\\
3586	19\\
3587	19\\
3588	19\\
3589	19\\
3590	19\\
3591	19\\
3592	19\\
3593	19\\
3594	19\\
3595	19\\
3596	19\\
3597	19\\
3598	19\\
3599	19\\
3600	19\\
3601	19\\
3602	19\\
3603	19\\
3604	19\\
3605	19\\
3606	19\\
3607	19\\
3608	19\\
3609	19\\
3610	19\\
3611	19\\
3612	19\\
3613	19\\
3614	19\\
3615	19\\
3616	19\\
3617	19\\
3618	19\\
3619	19\\
3620	19\\
3621	19\\
3622	19\\
3623	19\\
3624	19\\
3625	19\\
3626	19\\
3627	19\\
3628	19\\
3629	19\\
3630	19\\
3631	19\\
3632	19\\
3633	19\\
3634	19\\
3635	19\\
3636	19\\
3637	19\\
3638	19\\
3639	19\\
3640	19\\
3641	19\\
3642	19\\
3643	19\\
3644	19\\
3645	19\\
3646	19\\
3647	19\\
3648	19\\
3649	19\\
3650	19\\
3651	19\\
3652	19\\
3653	17\\
3654	17\\
3655	17\\
3656	17\\
3657	17\\
3658	17\\
3659	17\\
3660	17\\
3661	17\\
3662	17\\
3663	17\\
3664	17\\
3665	17\\
3666	17\\
3667	17\\
3668	17\\
3669	17\\
3670	17\\
3671	17\\
3672	17\\
3673	17\\
3674	17\\
3675	17\\
3676	17\\
3677	17\\
3678	17\\
3679	17\\
3680	17\\
3681	17\\
3682	17\\
3683	17\\
3684	17\\
3685	17\\
3686	17\\
3687	17\\
3688	17\\
3689	17\\
3690	17\\
3691	17\\
3692	17\\
3693	17\\
3694	17\\
3695	17\\
3696	17\\
3697	17\\
3698	17\\
3699	17\\
3700	17\\
3701	17\\
3702	17\\
3703	17\\
3704	17\\
3705	17\\
3706	17\\
3707	17\\
3708	17\\
3709	17\\
3710	17\\
3711	17\\
3712	17\\
3713	17\\
3714	17\\
3715	17\\
3716	17\\
3717	17\\
3718	17\\
3719	17\\
3720	17\\
3721	17\\
3722	17\\
3723	17\\
3724	17\\
3725	17\\
3726	17\\
3727	17\\
3728	17\\
3729	17\\
3730	17\\
3731	17\\
3732	17\\
3733	17\\
3734	17\\
3735	16\\
3736	16\\
3737	16\\
3738	16\\
3739	16\\
3740	16\\
3741	16\\
3742	16\\
3743	16\\
3744	16\\
3745	16\\
3746	16\\
3747	16\\
3748	16\\
3749	16\\
3750	16\\
3751	16\\
3752	16\\
3753	16\\
3754	16\\
3755	16\\
3756	16\\
3757	16\\
3758	16\\
3759	16\\
3760	16\\
3761	16\\
3762	16\\
3763	16\\
3764	16\\
3765	16\\
3766	16\\
3767	16\\
3768	16\\
3769	16\\
3770	16\\
3771	16\\
3772	16\\
3773	16\\
3774	16\\
3775	16\\
3776	16\\
3777	16\\
3778	16\\
3779	16\\
3780	16\\
3781	16\\
3782	16\\
3783	16\\
3784	16\\
3785	16\\
3786	16\\
3787	16\\
3788	16\\
3789	16\\
3790	16\\
3791	16\\
3792	16\\
3793	16\\
3794	16\\
3795	16\\
3796	16\\
3797	16\\
3798	16\\
3799	16\\
3800	16\\
3801	16\\
3802	16\\
3803	16\\
3804	16\\
3805	16\\
3806	16\\
3807	16\\
3808	16\\
3809	16\\
3810	16\\
3811	16\\
3812	16\\
3813	16\\
3814	16\\
3815	16\\
3816	16\\
3817	16\\
3818	16\\
3819	16\\
3820	16\\
3821	16\\
3822	16\\
3823	16\\
3824	16\\
3825	16\\
3826	16\\
3827	16\\
3828	16\\
3829	16\\
3830	16\\
3831	16\\
3832	16\\
3833	15\\
3834	15\\
3835	15\\
3836	15\\
3837	15\\
3838	15\\
3839	15\\
3840	15\\
3841	15\\
3842	15\\
3843	15\\
3844	15\\
3845	15\\
3846	15\\
3847	15\\
3848	15\\
3849	15\\
3850	15\\
3851	15\\
3852	15\\
3853	15\\
3854	15\\
3855	15\\
3856	15\\
3857	15\\
3858	15\\
3859	15\\
3860	15\\
3861	15\\
3862	15\\
3863	15\\
3864	15\\
3865	15\\
3866	15\\
3867	15\\
3868	15\\
3869	15\\
3870	15\\
3871	15\\
3872	15\\
3873	15\\
3874	14\\
3875	14\\
3876	14\\
3877	14\\
3878	14\\
3879	14\\
3880	14\\
3881	14\\
3882	14\\
3883	14\\
3884	14\\
3885	14\\
3886	14\\
3887	14\\
3888	14\\
3889	14\\
3890	14\\
3891	14\\
3892	14\\
3893	14\\
3894	14\\
3895	14\\
3896	14\\
3897	14\\
3898	14\\
3899	14\\
3900	14\\
3901	14\\
3902	14\\
3903	14\\
3904	14\\
3905	14\\
3906	14\\
3907	14\\
3908	14\\
3909	14\\
3910	14\\
3911	14\\
3912	14\\
3913	14\\
3914	14\\
3915	14\\
3916	14\\
3917	13\\
3918	13\\
3919	13\\
3920	13\\
3921	13\\
3922	13\\
3923	13\\
3924	13\\
3925	13\\
3926	13\\
3927	13\\
3928	13\\
3929	13\\
3930	13\\
3931	13\\
3932	13\\
3933	13\\
3934	13\\
3935	13\\
3936	13\\
3937	13\\
3938	13\\
3939	13\\
3940	13\\
3941	13\\
3942	13\\
3943	13\\
3944	13\\
3945	13\\
3946	13\\
3947	13\\
3948	13\\
3949	13\\
3950	13\\
3951	12\\
3952	12\\
3953	12\\
3954	12\\
3955	12\\
3956	12\\
3957	12\\
3958	12\\
3959	12\\
3960	12\\
3961	12\\
3962	12\\
3963	12\\
3964	12\\
3965	12\\
3966	12\\
3967	12\\
3968	12\\
3969	12\\
3970	12\\
3971	12\\
3972	12\\
3973	12\\
3974	12\\
3975	12\\
3976	12\\
3977	12\\
3978	12\\
3979	12\\
3980	12\\
3981	12\\
3982	12\\
3983	12\\
3984	12\\
3985	12\\
3986	12\\
3987	12\\
3988	12\\
3989	12\\
3990	12\\
3991	12\\
3992	12\\
3993	12\\
3994	12\\
3995	12\\
3996	12\\
3997	12\\
3998	12\\
3999	12\\
4000	12\\
4001	12\\
};
\addplot [color=mycolor1,solid,forget plot]
  table[row sep=crcr]{%
4001	12\\
4002	12\\
4003	12\\
4004	12\\
4005	12\\
4006	12\\
4007	12\\
4008	12\\
4009	12\\
4010	12\\
4011	12\\
4012	12\\
4013	12\\
4014	12\\
4015	12\\
4016	12\\
4017	12\\
4018	12\\
4019	12\\
4020	12\\
4021	12\\
4022	12\\
4023	12\\
4024	12\\
4025	12\\
4026	12\\
4027	12\\
4028	12\\
4029	12\\
4030	12\\
4031	12\\
4032	12\\
4033	12\\
4034	12\\
4035	12\\
4036	12\\
4037	12\\
4038	12\\
4039	12\\
4040	12\\
4041	12\\
4042	12\\
4043	12\\
4044	12\\
4045	12\\
4046	12\\
4047	12\\
4048	12\\
4049	12\\
4050	12\\
4051	12\\
4052	12\\
4053	12\\
4054	12\\
4055	12\\
4056	12\\
4057	12\\
4058	12\\
4059	12\\
4060	12\\
4061	12\\
4062	12\\
4063	12\\
4064	12\\
4065	12\\
4066	12\\
4067	12\\
4068	12\\
4069	12\\
4070	12\\
4071	12\\
4072	12\\
4073	12\\
4074	12\\
4075	12\\
4076	12\\
4077	12\\
4078	12\\
4079	12\\
4080	12\\
4081	12\\
4082	12\\
4083	12\\
4084	12\\
4085	12\\
4086	12\\
4087	12\\
4088	12\\
4089	12\\
4090	12\\
4091	12\\
4092	12\\
4093	12\\
4094	12\\
4095	12\\
4096	12\\
4097	12\\
4098	12\\
4099	12\\
4100	12\\
4101	12\\
4102	12\\
4103	12\\
4104	12\\
4105	12\\
4106	12\\
4107	12\\
4108	12\\
4109	12\\
4110	12\\
4111	12\\
4112	12\\
4113	12\\
4114	12\\
4115	12\\
4116	12\\
4117	12\\
4118	12\\
4119	12\\
4120	12\\
4121	12\\
4122	12\\
4123	12\\
4124	12\\
4125	12\\
4126	12\\
4127	12\\
4128	12\\
4129	12\\
4130	12\\
4131	12\\
4132	12\\
4133	12\\
4134	12\\
4135	12\\
4136	12\\
4137	12\\
4138	12\\
4139	12\\
4140	12\\
4141	12\\
4142	12\\
4143	12\\
4144	12\\
4145	12\\
4146	12\\
4147	12\\
4148	12\\
4149	12\\
4150	12\\
4151	12\\
4152	12\\
4153	12\\
4154	12\\
4155	12\\
4156	12\\
4157	12\\
4158	12\\
4159	11\\
4160	11\\
4161	11\\
4162	11\\
4163	11\\
4164	11\\
4165	11\\
4166	11\\
4167	11\\
4168	11\\
4169	11\\
4170	11\\
4171	11\\
4172	11\\
4173	11\\
4174	11\\
4175	11\\
4176	11\\
4177	11\\
4178	11\\
4179	11\\
4180	11\\
4181	11\\
4182	11\\
4183	11\\
4184	11\\
4185	11\\
4186	11\\
4187	11\\
4188	11\\
4189	11\\
4190	11\\
4191	11\\
4192	11\\
4193	11\\
4194	11\\
4195	11\\
4196	11\\
4197	11\\
4198	11\\
4199	11\\
4200	11\\
4201	11\\
4202	11\\
4203	11\\
4204	11\\
4205	11\\
4206	11\\
4207	11\\
4208	11\\
4209	11\\
4210	11\\
4211	11\\
4212	11\\
4213	11\\
4214	11\\
4215	11\\
4216	11\\
4217	11\\
4218	11\\
4219	11\\
4220	11\\
4221	11\\
4222	11\\
4223	11\\
4224	11\\
4225	11\\
4226	11\\
4227	11\\
4228	11\\
4229	11\\
4230	11\\
4231	11\\
4232	11\\
4233	11\\
4234	11\\
4235	10\\
4236	10\\
4237	10\\
4238	10\\
4239	10\\
4240	10\\
4241	10\\
4242	10\\
4243	10\\
4244	10\\
4245	10\\
4246	10\\
4247	10\\
4248	10\\
4249	10\\
4250	10\\
4251	10\\
4252	10\\
4253	10\\
4254	10\\
4255	10\\
4256	10\\
4257	10\\
4258	10\\
4259	10\\
4260	10\\
4261	10\\
4262	10\\
4263	10\\
4264	10\\
4265	10\\
4266	10\\
4267	10\\
4268	10\\
4269	10\\
4270	10\\
4271	10\\
4272	10\\
4273	10\\
4274	10\\
4275	10\\
4276	10\\
4277	10\\
4278	10\\
4279	10\\
4280	10\\
4281	10\\
4282	10\\
4283	10\\
4284	10\\
4285	10\\
4286	10\\
4287	10\\
4288	10\\
4289	10\\
4290	10\\
4291	10\\
4292	10\\
4293	10\\
4294	10\\
4295	10\\
4296	10\\
4297	10\\
4298	10\\
4299	10\\
4300	10\\
4301	10\\
4302	10\\
4303	10\\
4304	10\\
4305	10\\
4306	10\\
4307	10\\
4308	10\\
4309	10\\
4310	10\\
4311	10\\
4312	10\\
4313	10\\
4314	10\\
4315	10\\
4316	10\\
4317	10\\
4318	10\\
4319	10\\
4320	10\\
4321	10\\
4322	10\\
4323	10\\
4324	10\\
4325	10\\
4326	10\\
4327	10\\
4328	10\\
4329	10\\
4330	10\\
4331	10\\
4332	10\\
4333	10\\
4334	10\\
4335	10\\
4336	10\\
4337	10\\
4338	10\\
4339	10\\
4340	10\\
4341	10\\
4342	10\\
4343	10\\
4344	10\\
4345	10\\
4346	10\\
4347	10\\
4348	10\\
4349	10\\
4350	10\\
4351	10\\
4352	10\\
4353	10\\
4354	10\\
4355	10\\
4356	10\\
4357	10\\
4358	10\\
4359	10\\
4360	10\\
4361	10\\
4362	10\\
4363	10\\
4364	10\\
4365	10\\
4366	10\\
4367	10\\
4368	10\\
4369	10\\
4370	10\\
4371	10\\
4372	10\\
4373	10\\
4374	10\\
4375	10\\
4376	10\\
4377	10\\
4378	10\\
4379	10\\
4380	10\\
4381	10\\
4382	10\\
4383	10\\
4384	10\\
4385	10\\
4386	10\\
4387	10\\
4388	10\\
4389	10\\
4390	10\\
4391	10\\
4392	10\\
4393	10\\
4394	10\\
4395	10\\
4396	10\\
4397	10\\
4398	10\\
4399	10\\
4400	10\\
4401	10\\
4402	10\\
4403	10\\
4404	10\\
4405	10\\
4406	10\\
4407	10\\
4408	10\\
4409	10\\
4410	10\\
4411	10\\
4412	10\\
4413	10\\
4414	10\\
4415	10\\
4416	10\\
4417	10\\
4418	10\\
4419	10\\
4420	10\\
4421	10\\
4422	10\\
4423	10\\
4424	10\\
4425	10\\
4426	10\\
4427	10\\
4428	10\\
4429	10\\
4430	10\\
4431	10\\
4432	10\\
4433	10\\
4434	10\\
4435	10\\
4436	10\\
4437	10\\
4438	10\\
4439	10\\
4440	10\\
4441	10\\
4442	10\\
4443	10\\
4444	10\\
4445	10\\
4446	10\\
4447	10\\
4448	10\\
4449	10\\
4450	10\\
4451	10\\
4452	10\\
4453	10\\
4454	10\\
4455	10\\
4456	10\\
4457	10\\
4458	10\\
4459	10\\
4460	10\\
4461	10\\
4462	10\\
4463	10\\
4464	10\\
4465	10\\
4466	10\\
4467	10\\
4468	10\\
4469	10\\
4470	10\\
4471	10\\
4472	10\\
4473	10\\
4474	10\\
4475	10\\
4476	10\\
4477	10\\
4478	10\\
4479	10\\
4480	10\\
4481	10\\
4482	10\\
4483	10\\
4484	10\\
4485	10\\
4486	10\\
4487	11\\
4488	11\\
4489	11\\
4490	11\\
4491	11\\
4492	11\\
4493	11\\
4494	11\\
4495	11\\
4496	11\\
4497	11\\
4498	11\\
4499	11\\
4500	11\\
4501	11\\
4502	11\\
4503	11\\
4504	11\\
4505	11\\
4506	11\\
4507	11\\
4508	11\\
4509	11\\
4510	11\\
4511	11\\
4512	11\\
4513	11\\
4514	11\\
4515	11\\
4516	11\\
4517	11\\
4518	11\\
4519	11\\
4520	11\\
4521	11\\
4522	11\\
4523	11\\
4524	11\\
4525	11\\
4526	11\\
4527	11\\
4528	11\\
4529	11\\
4530	11\\
4531	11\\
4532	11\\
4533	11\\
4534	11\\
4535	11\\
4536	11\\
4537	11\\
4538	11\\
4539	11\\
4540	11\\
4541	11\\
4542	11\\
4543	11\\
4544	11\\
4545	11\\
4546	11\\
4547	11\\
4548	11\\
4549	11\\
4550	11\\
4551	11\\
4552	11\\
4553	11\\
4554	11\\
4555	11\\
4556	11\\
4557	11\\
4558	11\\
4559	11\\
4560	11\\
4561	11\\
4562	11\\
4563	11\\
4564	11\\
4565	11\\
4566	11\\
4567	11\\
4568	11\\
4569	11\\
4570	11\\
4571	11\\
4572	11\\
4573	11\\
4574	11\\
4575	11\\
4576	11\\
4577	11\\
4578	11\\
4579	11\\
4580	11\\
4581	11\\
4582	11\\
4583	11\\
4584	11\\
4585	11\\
4586	11\\
4587	11\\
4588	11\\
4589	11\\
4590	11\\
4591	11\\
4592	11\\
4593	11\\
4594	11\\
4595	11\\
4596	11\\
4597	11\\
4598	11\\
4599	11\\
4600	11\\
4601	11\\
4602	11\\
4603	11\\
4604	11\\
4605	11\\
4606	11\\
4607	11\\
4608	11\\
4609	11\\
4610	11\\
4611	11\\
4612	11\\
4613	11\\
4614	11\\
4615	11\\
4616	11\\
4617	11\\
4618	11\\
4619	11\\
4620	11\\
4621	11\\
4622	11\\
4623	11\\
4624	11\\
4625	11\\
4626	11\\
4627	11\\
4628	11\\
4629	11\\
4630	11\\
4631	11\\
4632	11\\
4633	11\\
4634	11\\
4635	11\\
4636	11\\
4637	11\\
4638	11\\
4639	11\\
4640	11\\
4641	11\\
4642	11\\
4643	11\\
4644	11\\
4645	11\\
4646	11\\
4647	11\\
4648	11\\
4649	11\\
4650	11\\
4651	11\\
4652	11\\
4653	11\\
4654	11\\
4655	11\\
4656	11\\
4657	11\\
4658	11\\
4659	11\\
4660	11\\
4661	11\\
4662	11\\
4663	11\\
4664	11\\
4665	11\\
4666	11\\
4667	11\\
4668	11\\
4669	11\\
4670	11\\
4671	11\\
4672	11\\
4673	11\\
4674	11\\
4675	11\\
4676	11\\
4677	11\\
4678	11\\
4679	11\\
4680	11\\
4681	11\\
4682	11\\
4683	11\\
4684	11\\
4685	11\\
4686	11\\
4687	11\\
4688	11\\
4689	11\\
4690	11\\
4691	11\\
4692	11\\
4693	11\\
4694	11\\
4695	11\\
4696	11\\
4697	11\\
4698	11\\
4699	11\\
4700	11\\
4701	11\\
4702	11\\
4703	11\\
4704	11\\
4705	11\\
4706	11\\
4707	11\\
4708	11\\
4709	11\\
4710	11\\
4711	11\\
4712	11\\
4713	11\\
4714	11\\
4715	11\\
4716	10\\
4717	10\\
4718	10\\
4719	10\\
4720	10\\
4721	10\\
4722	10\\
4723	10\\
4724	10\\
4725	10\\
4726	10\\
4727	10\\
4728	10\\
4729	10\\
4730	10\\
4731	10\\
4732	10\\
4733	10\\
4734	10\\
4735	10\\
4736	10\\
4737	10\\
4738	10\\
4739	10\\
4740	10\\
4741	10\\
4742	10\\
4743	10\\
4744	10\\
4745	10\\
4746	10\\
4747	10\\
4748	10\\
4749	10\\
4750	10\\
4751	10\\
4752	10\\
4753	10\\
4754	10\\
4755	10\\
4756	10\\
4757	10\\
4758	10\\
4759	10\\
4760	10\\
4761	10\\
4762	10\\
4763	10\\
4764	10\\
4765	10\\
4766	10\\
4767	10\\
4768	10\\
4769	10\\
4770	10\\
4771	10\\
4772	10\\
4773	10\\
4774	10\\
4775	10\\
4776	10\\
4777	10\\
4778	10\\
4779	10\\
4780	10\\
4781	10\\
4782	10\\
4783	10\\
4784	10\\
4785	10\\
4786	10\\
4787	10\\
4788	10\\
4789	10\\
4790	10\\
4791	10\\
4792	10\\
4793	10\\
4794	10\\
4795	10\\
4796	10\\
4797	10\\
4798	10\\
4799	10\\
4800	10\\
4801	10\\
4802	10\\
4803	10\\
4804	10\\
4805	10\\
4806	10\\
4807	10\\
4808	10\\
4809	10\\
4810	10\\
4811	10\\
4812	10\\
4813	10\\
4814	10\\
4815	10\\
4816	10\\
4817	10\\
4818	10\\
4819	10\\
4820	10\\
4821	10\\
4822	10\\
4823	10\\
4824	10\\
4825	10\\
4826	10\\
4827	10\\
4828	10\\
4829	10\\
4830	10\\
4831	10\\
4832	10\\
4833	10\\
4834	10\\
4835	10\\
4836	10\\
4837	10\\
4838	10\\
4839	10\\
4840	10\\
4841	10\\
4842	10\\
4843	10\\
4844	10\\
4845	10\\
4846	10\\
4847	10\\
4848	10\\
4849	10\\
4850	10\\
4851	10\\
4852	10\\
4853	10\\
4854	10\\
4855	10\\
4856	10\\
4857	10\\
4858	10\\
4859	10\\
4860	10\\
4861	10\\
4862	10\\
4863	10\\
4864	10\\
4865	10\\
4866	10\\
4867	10\\
4868	10\\
4869	10\\
4870	10\\
4871	10\\
4872	10\\
4873	10\\
4874	10\\
4875	10\\
4876	10\\
4877	10\\
4878	10\\
4879	10\\
4880	10\\
4881	10\\
4882	10\\
4883	10\\
4884	10\\
4885	10\\
4886	10\\
4887	10\\
4888	10\\
4889	10\\
4890	10\\
4891	10\\
4892	10\\
4893	10\\
4894	10\\
4895	10\\
4896	10\\
4897	10\\
4898	10\\
4899	10\\
4900	10\\
4901	10\\
4902	10\\
4903	10\\
4904	10\\
4905	10\\
4906	10\\
4907	10\\
4908	10\\
4909	10\\
4910	10\\
4911	10\\
4912	10\\
4913	10\\
4914	10\\
4915	10\\
4916	10\\
4917	10\\
4918	10\\
4919	10\\
4920	10\\
4921	10\\
4922	10\\
4923	10\\
4924	10\\
4925	10\\
4926	10\\
4927	10\\
4928	10\\
4929	10\\
4930	10\\
4931	10\\
4932	10\\
4933	10\\
4934	10\\
4935	10\\
4936	10\\
4937	10\\
4938	10\\
4939	10\\
4940	10\\
4941	10\\
4942	10\\
4943	10\\
4944	10\\
4945	10\\
4946	10\\
4947	10\\
4948	10\\
4949	10\\
4950	10\\
4951	10\\
4952	10\\
4953	10\\
4954	10\\
4955	10\\
4956	10\\
4957	10\\
4958	10\\
4959	10\\
4960	10\\
4961	10\\
4962	10\\
4963	10\\
4964	10\\
4965	10\\
4966	10\\
4967	10\\
4968	10\\
4969	10\\
4970	10\\
4971	10\\
4972	10\\
4973	10\\
4974	10\\
4975	10\\
4976	10\\
4977	10\\
4978	10\\
4979	10\\
4980	10\\
4981	10\\
4982	10\\
4983	10\\
4984	10\\
4985	10\\
4986	10\\
4987	10\\
4988	10\\
4989	10\\
4990	10\\
4991	9\\
4992	9\\
4993	9\\
4994	9\\
4995	9\\
4996	9\\
4997	9\\
4998	9\\
4999	9\\
5000	9\\
};
\end{axis}
\end{tikzpicture}%
	\vspace{-0.3cm}
	\caption{$H(x^t)$ as a function of time for $q=5$ and $c=5.$} 
	\label{energy}
\end{figure}\\
The trend of this curve matches our expectations since we constructed our procedure in order to minimize this energy.
Its behavior looks like a negative exponential: the energy abruptly drops at the beginning, while its decrease slows down as this value gets smaller.\\
These phases can be explained due to the randomness in the initial coloring which, generally, is very far from a proper coloring configuration.
In the first steps, we can easily recolor a chosen vertex in such a way that the number of neighbors having the same color immediately reduces.
However, as the algorithm progresses, it is less likely to pick a color for a given node that leads to a better configuration.
\newline\indent
We now proceed by plotting the average value of residual energy as a function of $c$ for $q$ equals to $3,5,7$.
\begin{figure}[h]
	\centering
	\setlength\figureheight{6cm} 		
	\setlength\figurewidth{0.8\textwidth}
	% This file was created by matlab2tikz.
% Minimal pgfplots version: 1.3
%
\definecolor{mycolor1}{rgb}{0.00000,0.44700,0.74100}%
\definecolor{mycolor2}{rgb}{0.85000,0.32500,0.09800}%
\definecolor{mycolor3}{rgb}{0.92900,0.69400,0.12500}%
%
\begin{tikzpicture}

\begin{axis}[%
width=0.95092\figurewidth,
height=\figureheight,
at={(0\figurewidth,0\figureheight)},
scale only axis,
xmin=0,
xmax=200,
xlabel={c},
xmajorgrids,
ymin=0,
ymax=30000,
ylabel={mean minimum residual energy},
ymajorgrids,
legend style={at={(0.05,0.75)},anchor=south west,legend cell align=left,align=left,draw=white!15!black}
]
\addplot [color=mycolor1,solid]
  table[row sep=crcr]{%
2	0\\
7	7.35\\
12	52.45\\
17	161.6\\
22	324.15\\
27	512.6\\
32	708.2\\
37	932.1\\
42	1160.5\\
47	1410.2\\
52	1643.1\\
57	1903.6\\
62	2158.1\\
67	2408.7\\
72	2697\\
77	2946.25\\
82	3235.9\\
87	3494.1\\
92	3795.5\\
97	4064.85\\
102	4324.95\\
107	4633.25\\
112	4902.45\\
117	5202.5\\
122	5505.75\\
127	5790.6\\
132	6102.7\\
137	6403.7\\
142	6708.55\\
147	7010.65\\
152	7310.1\\
157	7629.55\\
162	7916.4\\
167	8247\\
172	8551.7\\
177	8841.45\\
182	9159.2\\
187	9484\\
192	9782.8\\
197	10110.15\\
};
\addlegendentry{q=7};

\addplot [color=mycolor2,solid]
  table[row sep=crcr]{%
2	0.55\\
7	35.1\\
12	212.25\\
17	457.45\\
22	745.15\\
27	1078.95\\
32	1418.85\\
37	1776.35\\
42	2128.65\\
47	2515.4\\
52	2891.95\\
57	3268.5\\
62	3666.55\\
67	4073.6\\
72	4465.85\\
77	4864.15\\
82	5289.5\\
87	5707.4\\
92	6111.4\\
97	6539.4\\
102	6987.5\\
107	7376.9\\
112	7833.25\\
117	8245.5\\
122	8666.95\\
127	9090.4\\
132	9527.45\\
137	9953.6\\
142	10415.35\\
147	10881.9\\
152	11324.15\\
157	11753\\
162	12186.75\\
167	12610.45\\
172	13089.05\\
177	13511.5\\
182	13996.45\\
187	14407.6\\
192	14908.9\\
197	15310.2\\
};
\addlegendentry{q=5};

\addplot [color=mycolor3,solid]
  table[row sep=crcr]{%
2	9.9\\
7	331.7\\
12	846.2\\
17	1448.15\\
22	2069.1\\
27	2735.05\\
32	3410.25\\
37	4077.65\\
42	4782.35\\
47	5457.75\\
52	6219.8\\
57	6937.1\\
62	7638.75\\
67	8350.1\\
72	9122.95\\
77	9840.05\\
82	10592.65\\
87	11295.85\\
92	12057.35\\
97	12847.6\\
102	13580.85\\
107	14351.1\\
112	15149.55\\
117	15871.15\\
122	16662.65\\
127	17370.55\\
132	18120.9\\
137	18918.55\\
142	19696.9\\
147	20503.75\\
152	21258.2\\
157	22066.75\\
162	22749.45\\
167	23584.85\\
172	24354.55\\
177	25157.15\\
182	25956.4\\
187	26686.4\\
192	27494.9\\
197	28315.75\\
};
\addlegendentry{q=3};

\end{axis}
\end{tikzpicture}%
	\vspace{-0.3cm}
	\caption{$\langle H_{min}(q,c)\rangle$ as a function of $c$ for $q=3,\,5,\,7.$}
	\label{fig:botta}
\end{figure}\\
From this picture we can observe that as the parameter $c$ increases, also the mean energy is higher.
In fact, intuitively, as we increase the number of edges among nodes, it is more difficult to find a proper coloring in the graph.
Moreover, for a given value of $c$, the energy is higher for smaller $q$'s.
We can clearly deduce this behavior because it is easier to properly color the graph when the number of colors increases: the probability of finding a color that is different from the ones of its neighbors is larger.\\
In particular, we have always considered values of $c$ greater than one.
By doing so, we avoided considering trivial cases.
In fact, only for $c>1$ there is a giant connected component in the graph, thus the coloring task becomes difficult to perform.
On the other hand, when $c<1$, it is possible to implement appropriate algorithms which allow to almost surely find a proper coloring of a graph.
We have tried to test this trend by using our algorithm and the results are given in Figure \ref{fig:cpiccolo}.
\begin{figure}[h]
	\centering
	\setlength\figureheight{6cm} 		
	\setlength\figurewidth{0.8\textwidth}
	% This file was created by matlab2tikz.
% Minimal pgfplots version: 1.3
%
\definecolor{mycolor1}{rgb}{0.00000,0.8546,0.447}%
\definecolor{mycolor2}{rgb}{0.85000,0.32500,0.09800}%
\definecolor{mycolor3}{rgb}{0.92900,0.69400,0.12500}%
%
\begin{tikzpicture}

\begin{axis}[%
width=0.95092\figurewidth,
height=\figureheight,
at={(0\figurewidth,0\figureheight)},
scale only axis,
xmin=0,
xmax=2,
xlabel={c},
xmajorgrids,
ymin=0,
ymax=9,
ylabel={mean minimum residual energy},
ymajorgrids,
legend style={at={(0.05,0.75)},anchor=south west,legend cell align=left,align=left,draw=white!15!black}
]
\addplot [color=mycolor1,solid]
  table[row sep=crcr]{%
0	0\\
0.2	0\\
0.4	0\\
0.6	0\\
0.8	0\\
1	0\\
1.2	0\\
1.4	0\\
1.6	0\\
1.8	0\\
2	0\\
};
\addlegendentry{q=7};

\addplot [color=mycolor2,solid]
  table[row sep=crcr]{%
0	0\\
0.2	0\\
0.4	0\\
0.6	0\\
0.8	0\\
1	0\\
1.2	0\\
1.4	0\\
1.6	0\\
1.8	0.15\\
2	0\\
};
\addlegendentry{q=5};

\addplot [color=mycolor3,solid]
  table[row sep=crcr]{%
0	0\\
0.2	0\\
0.4	0\\
0.6	0.05\\
0.8	0.05\\
1	0.4\\
1.2	0.9\\
1.4	2.05\\
1.6	3.35\\
1.8	5.05\\
2	8.05\\
};
\addlegendentry{q=3};

\end{axis}
\end{tikzpicture}%
	\vspace{-0.3cm}
	\caption{$\langle H_{min}(q,c)\rangle$ as a function of $c<1$ for $q=3,\,5,\,7.$}
	\label{fig:cpiccolo}
\end{figure}\\
\indent

In this last part of the report, we illustrate the behavior of the other procedures that we have also analyzed.
In particular, the following cooling algorithms have been investigated:
\begin{itemize}
	\item $\beta = \beta - \log\left(\frac{\texttt{step}}{\texttt{maxIter}}\right)\cdotp\frac{\texttt{step}}{\texttt{maxIter}}\quad$ (schedule \textsl{log}),
	\item $\beta = \beta + 0.01\quad$ (schedule \textsl{lin}),
	\item $\beta = \beta\cdotp0.85\quad$ (schedule \textsl{pow}).
\end{itemize}
In the first two algorithms we update $\beta$ every 10 steps, while in the third one we update it every time the current iteration number assumes a value that is a power of 2.\\
The following figures are obtained choosing: \texttt{N=1000}, \texttt{q=7}, \texttt{c=5}, \texttt{$\beta$=5}, \texttt{maxIter=10000}.

\begin{figure}[h]
	\centering
	\setlength\figureheight{6cm} 		
	\setlength\figurewidth{1\textwidth}
	\input{picture/schedule_avg+var_q5_iter10000_new.tikz}
	\vspace{-0.3cm}
	\caption{$\langle H_{min}(q,c)\rangle$ and its variance as a function of $c$ for $q=5.$}
	\label{fig:schedules}
\end{figure}

%\begin{figure}[h]
%	\centering
%	\setlength\figureheight{6cm} 		
%	\setlength\figurewidth{0.8\textwidth}
%	\input{picture/schedule_tot_q5_iter10000.tikz}
%	\vspace{-0.3cm}
%	\caption{$\langle H_{min}(q,c)\rangle$ as a function of $c$ for $q=5$}
%	\label{fig:schedules}
%\end{figure}


From the picture on the left, we notice that, among all the chosen cooling procedures, none of them is significantly better than another, although the third one seems to be preferable. 
Probably, the number of nodes set for the simulation is too low to reveal a real difference.
However, it is not possible to establish what is the best procedure in an absolute sense; it could depend on the structure of the considered graph, $q$ or also on the number of iterations.\\
Furthermore, in addition to the average of the minimum residual energy, also its variance could play an important role as we can see from the plot on the right of Figure \ref{fig:schedules}. 
In fact, if this quantity were too large, then it could be more advantageous to run the algorithm more times but with considering a smaller number of iterations. 
On the other hand, a narrower variance leads us to increase the number of iterations in order to find the minimum value for the residual energy. 
%
% \begin{figure}
% 	\centering
% 	\subfigure[Prima figura]
%	\setlength\figureheight{6cm} 		
%	\setlength\figurewidth{0.5\textwidth}
%	\input{picture/schedule_tot_q5_iter10000.tikz}
% 	\hspace{5mm}
% 	\subfigure[Seconda figura]
% 	\centering
% 	\setlength\figureheight{6cm} 		
% 	\setlength\figurewidth{0.5\textwidth}
% 	\input{picture/schedule_variance_q5_iter10000.tikz}
% 	\caption{Titolo delle figure}
% \end{figure}